\documentclass[12pt]{article}

\usepackage{colortbl}
\usepackage{tabularx}
\usepackage{multirow}
\definecolor{Light}{gray}{0.80}
\definecolor{Dark}{gray}{0.20}
\usepackage{hyperref}
\usepackage{fancyhdr} 


\topmargin -0.50 in
\textheight 9.0 in
\oddsidemargin -0.25 in
\textwidth 6.75 in
\newcommand{\bl}[1]{\textcolor{blue}{#1}}
\newcommand{\re}[1]{\textcolor{red}{#1}}
\newcommand{\gr}[1]{\textcolor{cyan}{#1}}

\newcommand{\panel}[2]{\multicolumn{1}{|>{\columncolor[gray]
           {0.80}}#1|}{#2}}
\newcommand{\darkpanel}[1]{\multicolumn{1}{|>{\columncolor[gray]
           {0.40}}r|}{#1}}

\pagestyle{fancy}
\lhead{Western University, Astronomy 2022A, 2014}

\begin{document}

\thispagestyle{empty}

\begin{center}
\fcolorbox{black}{black}{\color{white}\Large\bf 
Astronomy 2022A: The Origin of the Universe}\\
\vspace{0.1in}
{\it Western University, Fall Term, 2014}
\end{center}

\vspace{0.0in}

\begin{description}

\item[\bf Calendar Description:] This course is designed for non-science students as an introduction to current ideas about the universe. Topics include the Big Bang, cosmic microwave background, origin of elements, and origin of galaxies, stars, and planetary systems.

\item[\bf Course Goals:] \hspace{1cm}   

At the end of this course, students will be able to:
\begin{itemize}
\item describe the the science of cosmology: its goals, methods and current state
\item identify and describe cosmology's current unanswered questions 
\item explain and use quantitative arguments in cosmology 
\end{itemize}
Please see the separate document ``Learning Outcomes for Astronomy 2022'' for 
detailed expectations and specific topics. 

\item[\bf Time and Location:] Monday/Wednesday 2:30--3:30 in Western Science Centre room 55.

\item[\bf Instructor:]  Prof.\ Pauline Barmby 
 \begin{description}

   \item[Office:] Physics \& Astronomy Building, Room  204 

   \item[Phone:] (519) 661--2111 ext.   81557 

   \item[E-mail:]  \verb$pbarmby$ [at] uwo [dot] ca\\ 
   We can be reached during the week through e-mail for simple inquiries, or to make an appointment, and will try to reply to e-mails within two working days of reception. \\ 
   
   \item[Office hours:]    \hspace{6cm}\\
   {\bf    Tues 1:00-4:00; Friday 2:30-4:30, or by appointment.} % TBD

   \end{description}

\item[\bf Textbook:]  {\it The State of the Universe}, P. G. Ferreira 2007, Phoenix.
A copy of the text is on reserve in Taylor Library (call number Q XX). % 
% custom course book

\item[\bf Website:] \verb$http://owl.uwo.ca$. % fill in something here
\newpage


\item[\bf Evaluation:] The grade assigned for this course will be based on: 
\begin{itemize}
\item final exam, 45\%
\item midterm exam, 22 \% 
\item assignments (best 2 of 3), 10\% each
\item participation via in-class exercises (4 of 5), 2\% each
\item participation via PeerWise, 5\%
\end{itemize}
Marks for all components will be posted on OWL. Any errors, or appeals to your scores, must be reported to 
the administrative TA (see contact information on OWL) within {\bf two weeks} of their initial posting.
The Department of Physics and Astronomy may, in exceptional cases, adjust the final course marks in order to conform to Departmental policy.

\item[\bf Tests and Exams:] There will be one midterm test plus a final exam. Each midterm or exam will generally test all the material taught up to that point. 
Question formats may include multiple-choice, short-answer, sentence answers, drawing diagrams or figures, or writing paragraphs;
there will not be extended essay-style questions. 
The midterm test will be held during class time, on {\bf  Wednesday October 15}, but may not be in the same classroom.
The final exam covers all of the course material and will be scheduled for the December exam period.

\item[\bf Assignments:] Short ($<$1500-word) written assignments will require students to summarize and evaluate articles, videos, or
other material relevant to the course. 

\item[\bf In-Class Exercises:] involve using the course workbook to step through 

\item[\bf PeerWise:]  ``is an online repository of multiple-choice questions that are created, answered, rated and discussed by students.''%
\footnote{\url{https://peerwise.cs.auckland.ac.nz}}. Registration is free. Students will be required to create 3 questions and answer
20 of their peers' questions over the course of the term; marks are based on completion and not correctness of questions or answers.

% edit
\item[\bf Other advice for successful performance] Slides shown in class are provided on the web but {\bf do not} include 
everything which might be tested. Some explanations given in class might not appear in the class notes but are testable material. 
To do well in this course, you must do the weekly readings. Some of the midterm and exam questions will come from the 
textbook but will not have been addressed in class.
\end{description}

\clearpage

\begin{center}
{\large \bf Astronomy 2022: Weekly Schedule, 2014}\\
\begin{tabular}{|l|l|l|l|} \hline
\panel{l}{{\bf Week}} & \panel{l}{{\bf Topic}} & \panel{l}{{\bf Text Chapter$^*$}} & \panel{l}{{\bf Notes}} \\
\panel{l}{{\bf starting}} & \panel{l}{} & \panel{l}{{}} & \panel{l}{} \\
\hline\hline
Sep.\ 8   & Measuring the universe & 1,2 &   \\
Sep.\ 15 &  Einstein \& spacetime & 3 & \\
Sep.\ 22 &  Evolving universe & 4& Assignment \#1 due\\
Sep.\ 29 &  Receding galaxies & 5 & in-class exercise \#1\\
Oct.\ 6   &   The hot Big Bang & 6 & in-class exercise \#5 \\
Oct.\ 13   &  Thanksgiving holiday (Monday) & &   \\
               & {\bf Midterm (Wednesday Oct.\ 15)} & & \\
Oct.\ 20 &   Making elements & 7, 8  & in-class exercise \#4\\
Oct.\ 27 &  Making matter & 9 & Assignment \#2 due\\ 
 &    {\em Oct 30/31: fall break} & & drop date Nov. 5 \\
Nov.\ 3 &  Gravity \& dark matter & 10, 11 & \\
Nov.\ 10 & Acceleration \& dark energy & 12, 13 & in-class exercise \#2 \\
Nov.\ 17 & The biggest picture & 14, 15& \\
Nov.\ 24 & Galaxy evolution & 16 & in-class exercise \#3\\
Dec.\ 1 &   In the beginning & 18 & Assignment \# 3 due\\
%
\hline
\rowcolor[gray]{0.8}
Dec. 6--17    &  (or later) {\bf Final Exam}      & & \\ \hline
\multicolumn{4}{l}{* For details on sections covered, see Learning Goals \& Outcomes}\\
\end{tabular}
\end{center}

Changes to this schedule will be discussed in class and posted on OWL. 

\clearpage

\begin{center}
{\large \bf Exciting Course Policies}\\
\end{center}

\begin{description}
\item[\bf Complaints and Suggestions:]
If you have a concern about something, please let us know. We rely on your feedback, and opportunities for anonymous feedback will be available. Please contact initially the person most directly concerned; this will usually be your instructor. If that is not satisfactory, or if there is something more general bothering you, talk it over with the Physics \& Astronomy Chair of Undergraduate Affairs or Department Chair (for contact information see http://www.physics.uwo.ca).

\item[\bf Classroom Conduct:]
Disruptive behaviour during lectures and tutorials will not be tolerated. This also includes the inappropriate use of mobile phones or other electronic devices. Please respect the rights of your classmates to benefit from class by limiting your conversations to those essential to the class. Students who persist in loud or rude behaviour will be asked to leave.

\item[\bf Electronic Devices:] No electronic devices will be allowed during tests and examinations.

\item[\bf Accessibility Statement:] Please contact the course instructor if you require material in an alternate format or if you require any other arrangements to make this course more accessible to you. You may also wish to contact Services for Students with Disabilities (SSD) at 661-2111 x 82147 for any specific question regarding an accommodation.

\item[\bf Registrarial and Support Services:] Student Support Services, Student Development Services and the Registrar are available online at 
\url{http://www.registrar.uwo.ca/} and/or at \url{http://westernusc.ca/services/}. Students who are in emotional/mental distress should refer to Mental Health\@Western 
\url{http://www.uwo.ca/uwocom/mentalhealth/} for a complete list of options about how to obtain help.

\item[\bf Missed assignments:] There is no grace period. You will receive a mark of zero. Only a serious medical or family excuse can override this. If this is the case, you should speak to your academic counsellor, who will confirm your excuse. No accommodations can be made without the approval of your academic counsellor.

\item[\bf Missed midterm test:] Documentation must be provided to the instructor in order for you to receive permission to write a make-up. This process should be begun by your bringing the documentation to your student counselling office. If you miss the make-up, again documentation must be provided, and your mark will be pro-rated.

\item[\bf Missed final exam:] Documentation must be provided to the academic counselors in your faculty in order for you to receive permission to write a make-up (usually scheduled the day following the end of the exam period: plan your travel accordingly!). If you miss the make-up, again documentation must be provided, and you will then write the exam at the next sitting of this course's final exam (typically one year later).

\item[\bf Illness or other serious circumstances:] If you are unable to meet a course requirement due to illness or other serious circumstances, you must provide valid medical or other supporting documentation to the Dean's office as soon as possible and contact your instructor immediately. It is the student's responsibility to make alternative arrangements with their instructor once the accommodation has been approved and the instructor has been informed. In the event of a missed final exam, a "Recommendation of Special Examination" form must be obtained from the Dean's Office immediately. For further information please see the medical section of the Academic Handbook.
\url{http://www.uwo.ca/univsec/handbook/appeals/accommodation_medical.pdf}
A student requiring academic accommodation due to illness should use the Student Medical Certificate when visiting an off-campus medical facility or request a Record's Release Form (located in the Dean's Office) for visits to Student Health Services. The form can be found at \url{http://www.uwo.ca/univsec/handbook/appeals/medicalform.pdf}
Students seeking academic accommodation on medical grounds for any missed tests, exams, participation components and/or assignments must apply to the Academic Counselling office of their home Faculty and provide documentation. Academic accommodation cannot be granted by the instructor or department.

\item[\bf Religious holidays:] A student who, due to unavoidable conflicts with religious holidays which (a) require an absence from the University or (b) prohibit or require certain activities (i.e., activities that would make it impossible for the student to satisfy the academic requirements scheduled on the day(s) involved), is unable to write examinations and term tests on a Sabbath or Holy Day in a particular term shall give notice of this fact in writing to his or her Dean as early as possible but not later than November 15th for mid-year examinations and March 1st for final examinations, i.e., approximately two weeks after the posting of the mid-year and final examination schedule respectively. In the case of mid-term tests, such notification is to be given in writing to the instructor within 48 hours of the announcement of the date of the mid-term test. The instructor(s) in the case of mid-term tests and the dean in the case of mid-year and spring final examinations will arrange for special examination(s) to be written at another time. In the case of mid-year and spring final examinations, the accommodation must occur no later than one month after the end of the examination period involved. It is mandatory that students seeking accommodations under this policy give notification before the deadlines, and that the Faculty accommodate these requests. The list of approved dates is given in the UWO calendar.

\item[\bf Academic misconduct: cheating] University policy states that cheating is a scholastic offence which can result in an academic penalty (which may include expulsion from the program). If you are caught cheating, there will be no second warning. Cheating includes having available any electronic devices other than a watch. You may not have a cell phone accessible during tests or exams, even to use it as a watch. Complete information on the University policies on academic offenses can be found in the Undergraduate section of this document \url{http://www.uwo.ca/univsec/handbook/appeals/scholoff.pdf}
Computer-marked multiple-choice tests and/or exams may be subject to submission for similarity review by software that will check for unusual coincidences in answer patterns that may indicate cheating.

\item[\bf Academic misconduct: plagiarism] Students must write their essays and assignments in their own words. Whenever students take an idea or a passage from another author, they must acknowledge their debt both by using quotation marks where appropriate and by proper referencing (such as footnotes or citations). Plagiarism is a major academic offence. For more details, see \url{http://www.uwo.ca/univsec/handbook/appeals/scholoff.pdf}
All required papers may be subject to submission for textual similarity review to the commercial plagiarism detection software under license to the University for the detection of plagiarism. All papers submitted will be included as source documents in the reference database for the purpose of detecting plagiarism of papers subsequently submitted to the system. Use of the service is subject to the licensing agreement, currently between The University of Western Ontario and Turnitin.com (\url{http://www.turnitin.com}).
\end{description}

{\em v0.1, \today}

\end{document}
