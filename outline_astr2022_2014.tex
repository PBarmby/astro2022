\documentclass[12pt]{article}

\usepackage{colortbl}
\usepackage{tabularx}
\usepackage{multirow}
\definecolor{Light}{gray}{0.80}
\definecolor{Dark}{gray}{0.20}
\usepackage{hyperref}
\usepackage{fancyhdr} 


\topmargin -0.50 in
\textheight 9.0 in
\oddsidemargin -0.25 in
\textwidth 6.75 in
\newcommand{\bl}[1]{\textcolor{blue}{#1}}
\newcommand{\re}[1]{\textcolor{red}{#1}}
\newcommand{\gr}[1]{\textcolor{cyan}{#1}}

\newcommand{\panel}[2]{\multicolumn{1}{|>{\columncolor[gray]
           {0.80}}#1|}{#2}}
\newcommand{\darkpanel}[1]{\multicolumn{1}{|>{\columncolor[gray]
           {0.40}}r|}{#1}}

\pagestyle{fancy}
\lhead{Western University, Astronomy 2022A, 2014}

\begin{document}

\thispagestyle{empty}

\begin{center}
\fcolorbox{black}{black}{\color{white}\Large\bf 
Astronomy 2022A: The Origin of the Universe}\\
\vspace{0.1in}
{\it Western University, Fall Term, 2014}
\end{center}

\vspace{0.0in}

\begin{description}

\item[\bf Calendar description:] This course is designed for non-science students as an introduction to current ideas about the universe. Topics include the Big Bang, cosmic microwave background, origin of elements, and origin of galaxies, stars, and planetary systems.

\item[\bf Course goals:] \hspace{1cm}   

At the end of this course, students will be able to:
\begin{itemize}
\item describe the science of cosmology and its relation to other fields of science
\item identify and describe cosmology's current unanswered questions 
\item explain how the scientific method and quantitative arguments are used in cosmology 
\end{itemize}
Please see the separate document ``Learning Outcomes for Astronomy 2022'' for 
detailed expectations and specific topics. 

\item[\bf Class time and location:] Monday/Wednesday 2:30--3:30 in Western Science Centre room 55.

\item[\bf Instructor:]  Prof.\ Pauline Barmby 
 \begin{description}

   \item[Office:] Physics \& Astronomy Building, Room  204 

   \item[Phone:] (519) 661--2111 ext.   81557 

   \item[E-mail:]  \verb$pbarmby [at] uwo [dot] ca $\\
   Please send e-mail from your Western e-mail account, and first check the course outline to see if your question is answered there. 
   I will try to reply to e-mails within two working days of reception. 
   
   \item[Instructor office hours:]
   {Mon/Wed 10:30-11:30, Friday 2:30-3:30, or by appointment. You don't have to have a serious question or problem to come to
   office hours; wanting to talk about the universe is fine!} 

  \item[TA office hours:] will be announced in class and on OWL.

   \end{description}

\item[\bf Textbooks:] (1) {\it The State of the Universe: A Primer In Modern Cosmology}, P. G. Ferreira. 2007, Phoenix.
(2) {\it Astronomy 2022, 2014--15 Custom workbook}, 2014, Pearson.\\
Copies of both are on reserve in Taylor Library. 

\item[\bf Website:] \url{http://owl.uwo.ca}. Class slides and course marks will be distributed through the course site
on OWL. OWL will also be used for assignment submission.

\item[\bf Social media:] The course hashtag is \#Astro2022.

\item[\bf Advice for successful performance:] to do well in this course, it is essential to read the relevant textbook chapters before
each week's classes. Slides shown in class supplement the textbook, assignments, and discussion but {\bf will not} alone provide
complete coverage of all learning objectives. Participating actively during class is important in developing
your understanding of the material. Scientific genius is {\it not} required for success in this course---but an open mind and a willingness
to learn will certainly help!

\newpage


\item[\bf Evaluation:] The grade assigned for this course will be based on: 
\begin{itemize}
\item final exam, 45\%
\item midterm exam, 22\% 
\item assignments (best 2 of 3), 10\% each (20\% total)
\item participation via in-class exercises (4 of 5), 2\% each (8\% total)
\item participation via PeerWise, 5\%
\end{itemize}
Marks for all components will be posted on OWL. So that we can avoid a rush at the end of term, please report any errors, or appeals to your scores, to 
the administrative TA (see contact information on OWL) within {\bf two weeks} of their initial posting.
The Department of Physics and Astronomy may, in exceptional cases, adjust the final course marks in order to conform to Departmental policy.

\item[\bf Tests and exams:] There will be one midterm test plus a final exam. 
Question formats may include multiple-choice, short-answer, sentence answers, drawing diagrams or figures, or writing paragraphs;
there will not be extended essay-style questions. 
The midterm test will be held during class time, on {\bf  Wednesday October 15}, but may not be in the same classroom.
The final exam covers all of the course material and will be scheduled for the December exam period.

\item[\bf Assignments:] Short ($<$1500-word) written assignments that may require students to summarize and evaluate articles, videos, or
other material relevant to the course; discuss or explain concepts in cosmology, or write well-justified `opinion pieces' on relevant topics.
Approximate due dates are in the schedule on the following page; exact due dates/times will be determined in consultation with the class
and posted on OWL.

\item[\bf In-class exercises:] involve using the course workbook to reason through complex concepts. Pre- or post-class homework
will be required but marked for completion only. Approximate due dates are in the schedule on the following page; exact due dates/times 
will be determined in consultation with the class and posted on OWL.

\item[\bf PeerWise:]  ``is an online repository of multiple-choice questions that are created, answered, rated and discussed by students.''%
\footnote{\url{https://peerwise.cs.auckland.ac.nz}}.  Students will be required to create 3 questions and answer
20 of their peers' questions over the course of the term; marks are based on completion and not correctness of questions or answers.
Registration is free. Instructions for how to create your PeerWise account and use the site are on OWL.

\end{description}

\clearpage


\begin{center}
{\large \bf Astronomy 2022A: 2014 Weekly Schedule}
\vspace{0.5cm}
\begin{tabular}{|l|l|l|l|} \hline
\panel{l}{{\bf Week}} & \panel{l}{{\bf Topic}} & \panel{l}{{\bf Text Chapter$^*$}} & \panel{l}{{\bf Notes}} \\
\panel{l}{{\bf starting}} & \panel{l}{} & \panel{l}{{}} & \panel{l}{} \\
\hline\hline
Sep.\ 8   & Measuring the universe & 2 &   \\
Sep.\ 15 &  Einstein \& spacetime & 3 & PeerWise intro, in class\\  
Sep.\ 22 &  Evolving universe & 4& Assignment \#1 due\\
Sep.\ 29 &  Receding galaxies & 5 & in-class exercise \#1\\
Oct.\ 6   &   The hot Big Bang & 6 & in-class exercise \#5 \\
Oct.\ 13   &  Thanksgiving holiday (Monday) & &   \\
\hline
               & {\bf Midterm (Wednesday Oct.\ 15)} & & \\
\hline               
Oct.\ 20 &   Making elements & 7  & in-class exercise \#4\\
Oct.\ 27 &  Making matter & 8, 9 & Assignment \#2 due\\ 
 &    {\em Oct. 30/31: fall break} & & {\em drop date Nov. 5 }\\
Nov.\ 3 &  Gravity \& dark matter & 10, 11 &  \\
Nov.\ 10 &  Acceleration&  12, workbook & in-class exercise \#2 \\
Nov.\ 17 &  Dark energy and structure & 13, 14&Assignment \# 3 due \\
Nov.\ 24 & Galaxies & 16 & in-class exercise \#3\\
Dec.\ 1 & Beginnings & 18, 1 & \\
%
\hline
\rowcolor[gray]{0.8}
Dec. 6--17    &  (TBD) {\bf Final Exam}      & cumulative&  \\ \hline
\multicolumn{4}{l}{*Refers to Chapters in {\em State of the Universe}. }\\
\end{tabular}
\end{center}
Exact dates for assignments and in-class exercises will be posted on OWL. 
Any changes to this schedule will be discussed in class and posted on OWL.

\vspace{1.5cm}

\begin{center}
{\large \bf Exciting Course and University Policies}\\
\end{center}

\begin{description}
\item[\bf Complaints and Suggestions:]
If you have a concern about something, please let us know. We rely on your feedback, and opportunities for anonymous feedback will be available. Please contact initially the person most directly concerned; this will usually be your instructor. If that is not satisfactory, or if there is something more general bothering you, talk it over with the Physics \& Astronomy Chair of Undergraduate Affairs or Department Chair (for contact information see \url{http://www.physics.uwo.ca}).

\item[\bf Classroom Conduct:]
Please respect the rights of your classmates to benefit from class by limiting your conversations to those essential to the class. 
Disruptive behaviour during class, including the inappropriate use of mobile phones or other electronic devices, will not be tolerated. 
Students who persist in such behaviour will be asked to leave.

\item[\bf Electronic Devices:] No electronic devices will be allowed during tests and examinations.

\item[\bf Accessibility Statement:] Please contact the course instructor if you require material in an alternate format or if you require any other arrangements to make this course more accessible to you. You may also wish to contact Services for Students with Disabilities (SSD) at 661-2111 x 82147 for any specific question regarding an accommodation.

\item[\bf Registrarial and Support Services:] Student Support Services, Student Development Services and the Registrar are available online at 
\url{http://www.registrar.uwo.ca/} and/or at \url{http://westernusc.ca/services/}. Students who are in emotional/mental distress should refer to Mental Health@Western 
\url{http://www.uwo.ca/uwocom/mentalhealth/} for a complete list of options about how to obtain help.

\item[\bf Accommodation for Medical Illness:] For UWO Policy on Accommodation for Medical Illness and a downloadable 
 Student Medical Certificate (SMC) see:
\url{http://www.uwo.ca/univsec/pdf/academic_policies/appeals/accommodation_medical.pdf}
[downloadable SMC: \url{https://studentservices.uwo.ca/} under the Medical Documentation heading].
Students seeking academic accommodation on medical grounds for any missed tests, exams,
participation components and/or assignments worth 10\% or more of their final grade must apply to
the Academic Counselling office of their home Faculty and provide documentation. Academic
accommodation cannot be granted by the instructor or department. Make-up exams are permitted
only with permission of Academic Counsellors and the instructor.

\item[\bf Accommodation for assignments/in-class exercises:] these activities are worth less than 10\% of the
final course grade and one of each can be missed without penalty; therefore no accommodation is necessary or expected
for the first missed activity. {\em For the same reason, late assignments will not be accepted.} If you are unable
to complete multiple activities due to illness or other serious circumstances, please see the instructions above 
and contact Academic Counselling.

\item[\bf Other academic accommodation:] The University has policies for accommodation for religious holidays and other circumstances,
listed at \url{http://www.uwo.ca/univsec/academic_policies/rights_responsibilities.html}. Please contact the instructor if any of these apply to you.

\item[\bf Academic offences:] Scholastic offences are taken seriously and students are directed to read the
appropriate policy, specifically, the definition of what constitutes a Scholastic Offence, at the
following Web site:
\url{http://www.uwo.ca/univsec/pdf/academic_policies/appeals/scholastic_discipline_undergrad.pdf}
Computer-marked multiple-choice tests and/or exams may be subject to submission for similarity review by software that will check for unusual coincidences in answer patterns that may indicate cheating.
All required papers may be subject to submission for textual similarity review to the commercial plagiarism detection software under license to the University for the detection of plagiarism. All papers submitted will be included as source documents in the reference database for the purpose of detecting plagiarism of papers subsequently submitted to the system. Use of the service is subject to the licensing agreement, currently between The University of Western Ontario and Turnitin.com (\url{http://www.turnitin.com}).
\end{description}

{\em v1.1, \today}

\end{document}
