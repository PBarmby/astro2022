\documentclass[11pt]{article}

\usepackage{ulem}

\topmargin -1.00 in
\textheight 9.5 in
\oddsidemargin -0.30 in
\textwidth 7.00 in

\begin{document}

\begin{center}
{\Large Course Goals \& Learning Outcomes for Astronomy 2022A, 2014}
\end{center}

\vspace{0.5cm}

{\bf Course goals}: At the end of this course, students will be able to:
\begin{itemize}
\item describe the science of cosmology: its goals, methods and current state
\item explain and use quantitative arguments in cosmology 
\item identify and describe cosmology's current unanswered questions 
\end{itemize}

\vspace{0.5cm}
{\bf Learning outcomes} for specific topics --- students will be able to:
\begin{enumerate}

\item Scale of the universe (Chapter 2, 12)
\begin{enumerate}
\item define and use the terms {\em star, planet,  galaxy, universe}
\item define light year, astronomical unit and relate these to the size of the above objects
\item do order-of-magnitude calculations with scientific notation 
\item compare human scales of space and time to astronomical scales 
\item define {\em parallax, Cepheid, supernova} and explain how these are used to measure distances 
\end{enumerate}

\item  Forces in the universe (Chapters 3, 9)
\begin{enumerate}
\item define {\em equivalence principle} and explain what it has to do with Einstein's theory of general relativity
\item explain how general relativity is used in understanding the structure and evolution of the universe
\item compare and contrast the possible types of {\em spacetime curvature} 
\item identify the 4 fundamental forces and describe their role in the past and present universe
\item explain the meaning of {\em quantum} as applied to forces in physics
\end{enumerate}

\item The expanding universe (Chapters 4, 5)
\begin{enumerate}
\item describe the {\em Cosmological Principle} and its consequences
\item explain the meaning of {\em scale factor}  and the different possible expansion histories for the universe
\item explain the effects of gravity, matter  and energy on the expansion
\item describe the observational evidence for the expanding universe 
\item define {\em redshift} and {\em Hubble constant}
\item explain Hubble's Law and interpret a Hubble diagram
\item describe the relation between the universe's expansion rate and its age 
\end{enumerate}

\item The early Universe (Chapter 6)
\begin{enumerate}
\item  explain what is meant by {\em Big Bang theory} and list some evidence for it 
\item define and use the terms {\em cosmic microwave background} and {\em recombination}
\item give the relationship between time, average energy, and temperature in the universe 
\item explain the concept of {\em thermal equilibrium} and its importance in cosmology
\item describe what happened (and approximate timescales) for the various eras in the early universe 
\end{enumerate}

\clearpage

\item Matter in the Universe (Chapters 7--9)
\begin{enumerate}
\item define {\em nucleus, element, isotope}
\item list and correctly order the most abundant elements in the universe and 
explain where and how these elements are formed
\item describe how the abundances of elements changes with time 
\item summarize the life cycle of stars
\item define {\em anti-matter} and explain why it is uncommon in the present-day universe
\end{enumerate}

\item Dark matter and dark energy (Chapters 10--13)
\begin{enumerate}
\item  define these two terms 
\item describe the constituents of the universe and their approximate proportions
\item list  and explain at least 2 pieces of observational evidence for dark matter and dark energy
\item compare and contrast the candidates for dark matter and dark energy, and the prospects for
directly detecting them
\item explain how the possible fates of the universe  relate to the matter and energy densities 
\item explain the relations between the {\em cosmological constant}, {\em vacuum energy} and {\em accelerating universe}
\end{enumerate}

\item Structure and scale (Chapters 14, 15)
\begin{enumerate}
\item define {\em homogeneity} and {\em isotropy} and their use in the context of cosmology
\item explain the idea of the {\em cosmological horizon}
\item define {\em inflation} in the cosmological context and explain what it means
\item explain how the cosmic microwave background relates to the distribution
of matter in the universe
\item explain what measurements of structure in the cosmic microwave background
tell us about the early universe
\end{enumerate}


\item Galaxies in the universe (Chapter 16)
\begin{enumerate}
\item describe the general size and shape of the {\em Milky Way} galaxy
\item list the different types of galaxies and their properties
\item explain the relationship between galaxies and {\em supermassive black holes}
\item explain the roles of gravity and angular momentum in the formation of galaxies and stars
\end{enumerate}

\item Beginning (Chapter 18)
\begin{enumerate}
\item explain what it means to talk about the `beginning' of the universe
\item define the idea of the {\em multiverse} and discuss its implications
\item explain what {\em quantum gravity} and {\em  string theory} have to do with the beginning of
the universe
\end{enumerate}

\end{enumerate}

\vspace{1cm}
\it{v 1.0, \today}

\end{document}