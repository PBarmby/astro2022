\documentclass[11pt]{article}

\usepackage{ulem}

\topmargin -1.00 in
\textheight 9.5 in
\oddsidemargin -0.30 in
\textwidth 7.00 in

\begin{document}

\begin{center}
{\Large Course Goals \& Learning Outcomes for Astronomy 2022A, 2014}
\end{center}

\vspace{0.5cm}

{\bf Course goals}: At the end of this course, students will be able to:
\begin{itemize}
\item describe the science of cosmology: its goals, methods and current state
\item explain and use quantitative arguments in cosmology 
\item identify and describe cosmology's current unanswered questions 
\end{itemize}

\vspace{0.5cm}
{\bf Learning outcomes} for specific topics --- students will be able to:
\begin{enumerate}

\item Scale of the universe (Chapter 2)
\begin{enumerate}
\item define and use the terms star, planet,  galaxy, universe
\item define light year, astronomical unit and relate these to the size of the above objects
\item do order-of-magnitude calculations with scientific notation 
\item compare human scales of space and time to astronomical scales 
%\item give approximate ages of astronomical objects, how measured 
%\item use the �cosmic calendar� to describe time scales in the universe
%\item describe and distinguish the components of large-scale structure in the universe (galaxies, groups, clusters, etc)
\end{enumerate}

\item  Forces in the universe (Chapters 3, 9)
\begin{enumerate}
\item describe the basic ideas of Einstein's theory of relativity
\item explain how general relativity is used in understanding the structure and evolution of the universe
\item compare and contrast the possible types of spacetime curvature 
\item identify the 4 fundamental forces and describe their role in the past and present universe
\item explain the meaning of {\em quantum} as applied to forces in physics
\end{enumerate}

\item The expanding Universe (Chapters 4, 5)
\begin{enumerate}
\item describe the {\em Cosmological Principle} and its consequences
\item list the different possible expansion histories for the universe
\item explain the effects of gravity, dark matter \& dark energy on the expansion
\item describe the observational evidence for the expanding universe 
\item explain Hubble's Law and interpret a Hubble diagram
\item describe the relation between the universe's expansion rate and its age 
\end{enumerate}

\item The early Universe (Chapters 6, 14, 15)
\begin{enumerate}
\item  explain what is meant by Big Bang theory and list some evidence for it 
\item define and use the terms {\rm cosmic microwave background} and {\em recombination}
\item give the relationship between time, average energy, and temperature in the early universe 
\item explain how the concept of {\em thermal equilibrium} is important in cosmology
\item describe what happened (and approximate timescales) for the various eras in the early universe 
\item list some of the open questions in the study of the early universe
\item 
\item
\end{enumerate}

\item Matter in the Universe (Chapters 7--9)
\begin{enumerate}
\item 
\item
\end{enumerate}

\item Dark matter and dark energy (Chapters 10--13)
\begin{enumerate}
\item  define these two terms 
\item describe the constituents of the universe (baryons w/ subdivisions, DM, DE) and their approximate proportions
\item list at least 2 pieces of evidence for dark matter
\item describe the evidence for dark energy 
\item list the candidates for each phenomenon and their prospects for detection
\item explain the possible fates of the universe and how these relate to the matter and energy densities 
\end{enumerate}

\item Galaxies in the universe (Chapter 16)
\begin{enumerate}
\item
\end{enumerate}



\item What does it all mean anyway? (Chapter 18)
\begin{enumerate}
\item describe the connections between Earth and the cosmos
\item describe the cosmological principle and the arguments both for and against it 
\item describe whether or how your personal perspective on the universe has changed as a result of the course % ??



\end{enumerate}

\item leftovers
\begin{enumerate}
\item explain the roles of gravity and angular momentum in the formation of stars, galaxies, etc
\item explain the nebular hypothesis and how it relates to solar system properties
\item distinguish between fission and fusion
\item explain what is meant by "cosmological test" and give examples
\item define Olbers' Paradox and explain its cosmological resolution 
\end{enumerate}

\end{enumerate}

\vspace{1cm}
\it{v 0.1, \today}

\end{document}